\functionalrequisite[id=08, importance=\veryhigh, urgency=\high]
{Registrar persona atendida}
{Emilio Cobos Álvarez, Celia Herrera Ferreira y Víctor Barrueco Gutiérrez}
{Enunciado de la práctica}
{OBJ-03 Gestionar emergencias}
{FRQ-02 Ver detalles de la emergencia, NFR-01 Privacidad, NFR-02 Seguridad de los datos, NFR-03 Escalabilidad, IRQ-03 Información de las emergencias, IRQ-4 Información acerca de la actuación de una dotación en una emergencia}
{El sistema deberá comportarse tal como se describe en el siguiente caso de uso cuando una persona externa haya sido atendida en una emergencia}
{La emergencia debe estar registrada y confirmada}
{
1 & El usuario selecciona una emergencia a la que añadir una persona atendida \\ \hline
2 & El sistema comprobará que la emergencia existe y que el usuario es responsable de alguna de las dotaciones asignadas a dicha emergencia \\ \hline
3 & Se introducen los datos de la persona atendida, se comprueban y queda registrada \\
}
{La persona queda marcada como atendida}
{
2 & Si la emergencia no existe o el usuario no está autorizado, el caso de uso termina \\ \hline
3 & Si la persona ya está marcada como atendida, el caso de uso termina \\ \hline
3 & Si no existe una persona atendida con esos datos, se da de alta \\
}

