\functionalrequisite[id=09, importance=\vital, urgency=\veryhigh]
{Escribir sección del informe}
{Emilio Cobos Álvarez, Celia Herrera Ferreira y Víctor Barrueco Gutiérrez}
{Enunciado de la práctica}
{OBJ-03 Gestionar emergencia}
{FRQ-02 Ver detalles de la emergencia, NFR-01 Privacidad, NFR-02 Seguridad de los datos, NFR-03 Escalabilidad, IRQ-01 Información de usuario, IRQ-02 Información de las dotaciones, IRQ-03 Información de las emergencias, IRQ-04 Información acerca de la actuación de una dotación en una emergencia, IRQ-05 Información del inventario de un determinado centro, IRQ-06 Información de los vehículos, IRQ-08 Información de los comentarios}
{El sistema deberá comportarse tal como se describe en el siguiente caso de uso cuando un responsable de una dotación quiera escribir una sección del informe de una emergencia a la que su dotación haya atendido}
{La emergencia debe estar registrada y resuelta, el usuario debe ser responsable de una dotación asignada a la emergencia}
{
1 & El usuario elige una emergencia para escribir su informe \\ \hline
2 & El sistema comprobará que la emergencia está resuelta, que el usuario es un responsable de alguna de las dotaciones asignadas a la emergencia, y que no hay una sección del informe previamente asociada a su dotación para esa emergencia \\ \hline
3 & El usuario escribe la sección del informe, indicando el material utilizado \\ \hline
4 & La sección queda registrada \\
}
{La sección queda correctamente añadida al informe de la emergencia}
{
2 & Si alguna de las condiciones especificadas no se cumple, el caso de uso termina \\ \hline
4 & Si la sección es la última entregada (hay tantas secciones como dotaciones asignadas) la emergencia se marca como finalizada, y se envía una notificación a todos los implicados en ella \\
}

