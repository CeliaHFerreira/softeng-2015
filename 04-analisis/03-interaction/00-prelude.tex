\subsection{Vista de interacción}
Se han hecho un total de 12 diagramas (tres en actualizar inventario y otros tres en gestionar dotaciones). No se han hecho más por limitaciones temporales.

\subsubsection{Comentar emergencia}
Este es el único diagrama en el que se hace efectivamente una comprobación para ver si la emergencia escogida está asignada. En otros diagramas se pasa directamente \texttt{assigned\_emergency} por simplicidad.

Igualmente, sólo se detalla el caso en el que las aserciones se cumplen correctas, si se quisiera detallar más el caso de uso, las correspondientes llamadas devolverían \texttt{Err(Reason)}, que sería propagado hacia el usuario, terminando el caso de uso.

\includegraphics{diagrams/interaction/comment-emergency.dia.png}

\subsubsection{Reportar emergencia}
\includegraphics{diagrams/interaction/report-emergency.dia.png}

\subsubsection{Confirmar emergencia}
\includegraphics{diagrams/interaction/confirm-emergency.dia.png}

\subsubsection{Ver detalles de emergencia}
\includegraphics{diagrams/interaction/emergency-details.dia.png}

\subsubsection{Marcar emergencia como resuelta}
\includegraphics{diagrams/interaction/mark-emergency-as-resolved.dia.png}

\subsubsection{Registrar persona atendida}
\includegraphics{diagrams/interaction/register-attended-person.dia.png}

\subsubsection{Actualizar inventario}
\includegraphics{diagrams/interaction/update-inventory.dia.png}

\subsubsection{Gestionar dotaciones}
\includegraphics{diagrams/interaction/manage-dotations.dia.png}

